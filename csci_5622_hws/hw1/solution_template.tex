\documentclass[11pt]{article}

% ==== PACKAGES ==== %
% \usepackage{fullpage}
\usepackage{amsmath,amssymb,amsthm}
\usepackage{epic}
\usepackage{eepic}
\usepackage{hyperref}
\usepackage{listings}
\usepackage{float}
\usepackage{graphicx}
\usepackage{fancyhdr}
\usepackage{color}
\usepackage[letterpaper, margin=1in]{geometry}

% ==== MARGINS ==== %
% \pagestyle{empty}
% \setlength{\oddsidemargin}{0in}
% \setlength{\textwidth}{6.8in}
% \setlength{\textheight}{9.5in}

\pagestyle{fancy}
\fancyhf{}
\rhead{CSCI 5622}
\lhead{Homework 1}
\rfoot{Page \thepage}


\newtheorem*{solution*}{Solution}
\newtheorem{lemma}{Lemma}[section]
\newtheorem{theorem}[lemma]{Theorem}
\newtheorem{claim}[lemma]{Claim}
\newtheorem{definition}[lemma]{Definition}
\newtheorem{corollary}[lemma]{Corollary}
\lstset{moredelim=[is][\bfseries]{[*}{*]}}

% ==== DOCUMENT PROPER ==== %
\begin{document}

\thispagestyle{empty}

% --- Header Box --- %
\newlength{\boxlength}\setlength{\boxlength}{\textwidth}
\addtolength{\boxlength}{-4mm}

\begin{center}\framebox{\parbox{\boxlength}{\bf
      Machine Learning \hfill Homework 1\\
      CSCI 5622 Fall 2017 \hfill Due Time Sep 15, 2017\\
      Name: \textcolor{red}{Chi Chen} \hfill CU identitykey: \textcolor{red}{chch4713}
}}
\end{center}




\section{K-nearest Neighbor (40pts)}

\begin{solution*}
	1. What is the role of the number of training instances to accuracy (hint: try different “--limit” and plot accuracy vs. number of training instances)?
	2. What numbers get confused with each other most easily?
	3. What is the role of k to training accuracy?
	4. In general, does a small value for k cause “overfitting” or “underfitting”?
\end{solution*}


\section{Cross Validation (30pts)}

\begin{solution*}
	1. What is the best k chosen from 5-fold cross validation with “--limit 500”? 
	2. What is the best k chosen from 5-fold cross validation “--limit 5000”?
	3. Is the best k consistent with the best performance k in problem 1?
\end{solution*}

\section{Bias-variance tradeoff (20pts)}

\begin{solution*}
	
\end{solution*}


\end{document}
